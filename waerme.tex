% coding:utf-8

%----------------------------------------
%FOSAPHY, a LaTeX-Code for a summary of basic physics
%Copyright (C) 2013, Mario Felder

%This program is free software; you can redistribute it and/or
%modify it under the terms of the GNU General Public License
%as published by the Free Software Foundation; either version 2
%of the License, or (at your option) any later version.

%This program is distributed in the hope that it will be useful,
%but WITHOUT ANY WARRANTY; without even the implied warranty of
%MERCHANTABILITY or FITNESS FOR A PARTICULAR PURPOSE.  See the
%GNU General Public License for more details.
%----------------------------------------

\chapter{Wärme}

\section{Konstanten}
\[
\boxed{\begin{aligned}	
		&Avogadrozahl \\
		&N_{A} = 6.00221 \cdot 10^{23} Teilchen\\
		\\
		&Universelle Gaskonstante\\
		&R = 8.314472 \frac{J}{mol \cdot K}\\
		\\
		&Boltzmann\\
		&k_{B} = \frac{R}{N_{A}}=1.381 \cdot 10^{23} \frac{J}{K}
	\end{aligned}}	\]

\subsection{Ideale Gasgleichung}

\[pV = nRT = \frac{N_{tot}}{N_{A}}RT = \frac{m_{tot}}{M}RT\]

\begin{tabular*}{\linewidth}{p{0.15\linewidth}lp{0.37\linewidth}}
	\textbf{Variable}				&	\textbf{Bedeutung}		& \textbf{Einheit}\\
	\hline
	\rowcolor{white}p			&      Gasdruck				&$Pa$\\
	\rowcolor{lgray}T			&	Gastemperatur			& $K$\\
	\rowcolor{white}$N_{tot}$	&	Anzahl Moleküle im Gas	&$ $\\				
	\rowcolor{lgray}$m_{tot}$	&	Gasmasse				&$ $\\
	\rowcolor{white}n			&	Anzahl mol im Gas		&$ $\\
	\rowcolor{lgray}$N_{A}$		&	Avogadrozahl			&$N_{A} = 6.022 \cdot 10^{23} \frac{Teilchen}{mol}$\\	
	\rowcolor{white}M			&	Molmasse				&$ $\\
\end{tabular*}


\section{Luftdruck vs. H\"ohe bei konstanter Temperatur}
Der Schweredruck in einm Fluid ist $\Delta p=-\rho \cdot g\Delta y$.\\

\[
\boxed{\begin{aligned}	
		p(y)&=p_0 \cdot \e^{- \frac{m_{mol}g}{RT}y} = p_0 \e^{-\frac{y}{H}}
		\\
		H&= \frac{RT}{m_{mol}g}
	\end{aligned}}\]
\newline


\subsection{Energie}

Kinetische Energie im idealen Gas. \newline
	\[ E_{Gas} =\frac{3}{2}nRT\]
\newline

Mittlere kinetische Energie eines Moleküls im idealen Gas.\newline
	\[ \langle E_{kin, k\"ul} \rangle = \frac{1}{2}m_{k\"ul} \langle v^2 \rangle = \frac{3}{2}k_{B}T\]
\newline



\subsection{Geschwindigkeiten}
\[
	v_{rms}=sqrt{\langle v^2 \rangle}=sqrt{\frac{3k_{B}T}{m_{k\"ul}}}=sqrt{\frac{3RT}{m_{mol}}}
\]
\\
Wahrscheinlichste Geschwindigkeit:
\[
	v_{w}=sqrt{\frac{2k_{B}T}{m_{k\"ul}}}=sqrt{\frac{2RT}{m_{mol}}}
\]
\\
Durchschnittsgeschwindigkeit:
\[
	v_{av}=sqrt{\frac{8k_{B}T}{\pi m_{k\"ul}}}=sqrt{\frac{8RT}{\pi m_{mol}}}
\]
\newline


\subsection{Freiheitsgrade FG}
Das einatomibe, ideale Gas hat genau drei Bewegungs-Freiheitgrade: links nach rechts, hinten nach vorn, unten nach oben. Zweiatomige Gase haben mehr Bewegungsm\"oglichkeiten.

Äquipartitions Gesetz der klassischen Mechanik: 
Auf jeden aktiven Freiheitsgrad eines Moleküls in einem Gas der Temperatur T entfällt im Mittel die Energie $\frac{1}{2}k_{B}T$.

\[\boxed{\begin{aligned}
		\frac{\langle E_{kül} \rangle}{FG}= \frac{1}{2}k_{B}T
		\\
		\frac{\langle E_{mol} \rangle}{FG}= \frac{1}{2}R \cdot T
	\end{aligned}}\]	

\subsection{Spezifische Wärmekapazität c}
Um die Temperatur einer Substanz zu erhöhen, kann man ihr W\"arme Q zuf\"uhren. W\"arme ist eine Energieform [J]. Eine Kalorie entspricht 4.186J.

\[\boxed{\begin{aligned}	
		Q=m \cdot c \cdot \Delta T \rightarrow dQ0m \cdot c \cdot dT
		\\
		c pro Masse: c_{(m)}=\frac{1 \cdot dQ}{m \cdot  dT}
		\\
		c pro Mol: c_{(n)}=\frac{1 \cdot dQ}{n \cdot  dT}		
		\\
		c_{(m)}=\frac{c_{(n)}}{m_{mol}}
	\end{aligned}}\]\newline
	
Spezifische Wärmekapazität des \textbf{idealen Gases} pro Mol, bei $\underline{\mathrm{konstantem}}$ Gasvolumen.

		Einatomigen Gases:
		\[\boxed{			
			c_{(n)V}=c_{V}=\frac{3}{2}R
		}\]\newline
		Zweiatomigen Gases:
		\[\boxed{	
			c_{(n)V}=c_{V}=\frac{5}{2}R							
		}\]\newline

\subsection{Mittlere freie Weglänge $\Lambda$}
Wir betrachten die Moleküle alas harte Kugeln mit Radius $r$ und leiten eine Kollisionszeit $t_{mean}$ und eine mittlere freie Weglänge $\Lambda$ her.

Mittlere Kollisionszeit
\[ t_{mean} = \frac{dt}{dN}= \frac{V}{4 \pi \sqrt{2}r^2vN} \]
	
Mittlere freie Wegl\"ange
\[ \Lambda =v \cdot t= \frac{V}{4 \pi \sqrt{2}r^2 \cdot N}=  \frac{k_{B} \cdot T}{4 \pi \sqrt{2}r^2 \cdot p} \]

\subsection{Wahrscheinlichkeit}
Die Verteilfunktion der molekularen Geschwindigkeiten f(v) kann mittels statistischer Mechanik hergeleitet werden.

Maxwell-Boltzmann Verteilung
\[ f(v)=4\pi \left(\frac{m_{mol}}{2\pi \cdot RT}\right)^{\frac32} \cdot v^2 \cdot \e^{-\frac{m_{mol} \cdot v^2}{2RT}}\]

\[ f(v)= 4\pi \left(\frac{m_{kul}}{2\pi \cdot k_{B}T}\right)^{\frac32} \cdot v^2 \cdot \e^{-\frac{m_{kul} \cdot v^2}{2k_{B}T}}\]

Wahrscheinlichkeitsdichte:
\[ W(v_{1},v_{2})= \int\limits_{v_{1}}^{v_{2}} f(v)\di v\]

\[ W(v_{1},v_{2})= \int\limits_{v_{1}}^{v_{2}} 4\pi \left(\frac{m_{mol}}{2\pi \cdot RT}\right)^{\frac32} \cdot v^2 \cdot \e^{-\frac{m_{mol} \cdot v^2}{2RT}}\di v\]
