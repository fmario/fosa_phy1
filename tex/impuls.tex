% coding:utf-8

%----------------------------------------
%FOSAPHY, a LaTeX-Code for a summary of basic physics
%Copyright (C) 2013, Mario Felder

%This program is free software; you can redistribute it and/or
%modify it under the terms of the GNU General Public License
%as published by the Free Software Foundation; either version 2
%of the License, or (at your option) any later version.

%This program is distributed in the hope that it will be useful,
%but WITHOUT ANY WARRANTY; without even the implied warranty of
%MERCHANTABILITY or FITNESS FOR A PARTICULAR PURPOSE.  See the
%GNU General Public License for more details.
%----------------------------------------

\chapter{Impuls und Kraftstoss}
\section{Impuls $\vec{p}$}
Definition:
\[
	\boxed{
		\vec{p}=m\cdot\vec{v}
	}
\]\\
Das zweite Newtonsche Gesetz verallgemeinert:
\[
	\boxed{
		\sum{\vec{F}}=\frac{\mathrm{d}\vec{p}}{\mathrm{d}t}
	}
\]

\section{Kraftstoss $\vec{J}$}
Definition des Kraftstosses (Impuls\"anderung):
\[
	\boxed{
		\vec{J}=\int_{t_1}^{t_2}{(\sum{\vec{F}})\mathrm{d}t}=\vec{p}_2-\vec{p}_1
	}
\]\\
Durchschnittliche Kraft:
\[
	\boxed{
		\vec{F}_{average}=\frac{1}{\Delta t}\int_{t_1}^{t_2}{\vec{F}(t)\mathrm{d}t}=\frac{\vec{J}}{\Delta t}
	}
\]

\section{Impulserhaltung}
Definition Gesamtimpuls:
\[
	\boxed{
		\vec{P}=\vec{p}_A+\vec{p}_B+\vec{p}_C+\ldots
	}
\]
Ist die Vektrosumme aller \"ausseren Kr\"afte auf ein System Null, bleibt der Gesamtimpuls erhalten:
\[
	\boxed{
		\sum{\vec{F}_{extern}}=0\leftrightarrow\vec{P}=konst.
	}
\]
\subsection{elastischer Stoss}
\textit{Definition:}\\
Beim elastischen Stoss bleibt die kinetische Energie vor und nach dem Stoss vollst\"andig erhalten.\\
\newline
Mit dem Energie- und Impulserhalt:
\[\begin{split}
	\frac{1}{2}m_Av_{A1}^2+\frac{1}{2}m_Bv_{B1}^2=\frac{1}{2}m_Av_{A2}^2+\frac{1}{2}m_Bv_{B2}^2\\
	m_A\vec{v}_{A1}+m_B\vec{v}_{B1}=m_A\vec{v}_{A2}+m_B\vec{v}_{B2}
\end{split}\]\\
ergibt sich:
\[
	\boxed{\begin{split}
		v_{A2}=\frac{m_A\cdot v_{A1}+m_B(2\cdot v_{B1}-v_{A1})}{m_A+m_B}\\
		v_{B2}=\frac{m_B\cdot v_{B1}+m_A(2\cdot v_{A1}-v_{B1})}{m_A+m_B}
	\end{split}}
\]

\subsection{inelastischer Stoss}
\textit{Definition:}\\
Beim inelastischen Stoss wird ein Teil der kinetischen Energie in Verformungsarbeit gesteckt.\\
\newline
Gemeinsame Geschwindigkeit nach dem Stoss:
\[
	\vec{v}_{A2}=\vec{v}_{B2}=\vec{v}_{2}
\]
\[
	\boxed{
		m_A\vec{v}_{A1}+m_B\vec{v}_{B1}=(m_A+m_B)\cdot\vec{v}_2
	}
\]\\
$\vec{v}_2$ ist die Geschwindigkeit des Schwerpunktes der beiden K\"orper.