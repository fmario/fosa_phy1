% coding:utf-8

%----------------------------------------
%FOSAPHY, a LaTeX-Code for a summary of basic physics
%Copyright (C) 2013, Mario Felder

%This program is free software; you can redistribute it and/or
%modify it under the terms of the GNU General Public License
%as published by the Free Software Foundation; either version 2
%of the License, or (at your option) any later version.

%This program is distributed in the hope that it will be useful,
%but WITHOUT ANY WARRANTY; without even the implied warranty of
%MERCHANTABILITY or FITNESS FOR A PARTICULAR PURPOSE.  See the
%GNU General Public License for more details.
%----------------------------------------

\chapter{Kraft}

Die 4 fundamentalen K\"aften sind:\\
\fbox{\parbox{\linewidth}{
	\begin{itemize}
		\item \textbf{Gravitationskraft} (Anziehung zwischen Massen)\newline
			(Bsp: Anziehung zwischen Sonne und Erde, Gezeitenkr\"afte)
		\item \textbf{Elektromagnetische Kraft} (Kr\"afte zwischen Ladungen)\newline
			(Bsp: Reibung, Seilkraft, Lorentzkraft)
		\item \textbf{Schwache Kraft und starke Kraft} (Kernkr\"afte)\newline
			(Bsp: Radioaktiver Zerfall, Anziehung zwischen Protonen und Neutronen)
	\end{itemize}
}}
\newline
\fbox{\parbox{\linewidth}{
	\begin{itemize}
		\item \textbf{Kr\"afte sind Vektoren:}\newline
			$\vec{F}_{Res}=\vec{F}_1+\vec{F}_2+\vec{F}_3+\ldots$\newline
			$\vec{F}=(F_x,F_y,F_z)=(F_r,F_\varphi,F_z)$
		\item \textbf{Tr\"agheitsgesetz} (1. Axiom)\newline
			$\vec{F}_{Res}=0\leftrightarrow\vec{a}=0\leftrightarrow\vec{v}=\mathrm{konstant}$
		\item \textbf{Bewegungsgleichung} (2. Axiom)\newline
			$\vec{F}_{Res}=m\cdot\vec{a}\leftrightarrow F_x=m\cdot a_x,F_y=m\cdot a_y,F_z=m\cdot a_z$
		\item \textbf{Wechselwirkungsgesetz} (3. Axiom)\newline
			$\vec{F}_{K\"orper\;A\;auf\;K\"orper\;B}=-\vec{F}_{K\"orper\;B\;auf\;K\"orper\;A}$
	\end{itemize}
}}

\section{\"Ubersicht}

\begin{footnotesize}
\begin{tabular*}{\linewidth}{|p{0.17\linewidth}lp{0.35\linewidth}|}
		\hline
		\textbf{Kraft}				&	\textbf{Gleichung}				& \textbf{Ursprung und Bemerkung}\\\hline
		\rowcolor{white}Feder									& $F_{Feder}=k\cdot x$   $(\vec{F}_H=-k\cdot\vec{x})$	& (em); lineare N\"aherung - Hooke'sches Gesetz\\
		\rowcolor{lgray}Normalkraft						& $F_N=F_g\cdot\cos\theta$	& (em); $F_N$ ist immer senkrecht zur Kontaktfl\"ache\\
		\rowcolor{white}Hangkraft							& $F_H=m\cdot g\cdot\sin\theta$ & (em); $F_H$ ist immer parallel zu Kontaktfl\"ache\\
		\rowcolor{lgray}Haftreibung						& $F_{HR}<F_{HR,max}=\mu_{HR}\cdot F_N$ & (em); Parallel zur Kontaktfl\"ache und der angreifenden Kraft engegengesetzt\\
		\rowcolor{white}Gleitreibung					& $F_{GR}=\mu_{GR}\cdot F_N$& (em); Der Bewegung entgegengesetzt; Van der Waals Kr\"afte\\
		\rowcolor{lgray}Lorentzkraft					& $F_L=qvB\cdot\sin\theta$	& (em); $\vec{F}_L=q(\vec{v}\times\vec{B})$\\
		\rowcolor{white}Hydrostatische Kraft	& $F_{hydr}=\rho gh\cdot A=\int\rho gh\mathrm{d}A$	& Gravitation (und em); $p_{hydr}=\rho gh$ ist der hydrostatische Druck\\
		\rowcolor{lgray}Auftrieb							& $F_A=\rho_{Fluid}gV_{K\"orper}$ & Gravitation (und em)\\
		\hline
\end{tabular*}
\end{footnotesize}
\newline

\section{Federkraft}

\textbf{Hooke'sches Gesetz:} \\
$\boxed{F_{Feder}=k\cdot \Delta x }$   $\left[k\right]=\frac{N}{m}$ \\
\\
\textbf{Federn in Serie:}
\[
	F_{Res}=k_{Res}\cdot\Delta x
\]
wobei:
\[
	\boxed{k_{Res}=\frac{k_1\cdot k_2}{k_1+k_2}}
\]
oder allgemein:
\[
	\boxed{k_{Res}=\frac{1}{\frac{1}{k_1}+\frac{1}{k_2}+\frac{1}{k_3}+\ldots}}
\]
\\
\textbf{Federn parallel:}
\[
	\boxed{F_{Res}=k_{Res}\cdot\Delta x=F_{H,1}+F_{H,2}+\ldots=(k_1+k_2+\ldots)\cdot\Delta x}
\]

\section{Reibung}
\subsection{Kontaktkr\"afte: Normal- \& Reibungskraft}

\fbox{\parbox{\linewidth}{Die \textbf{Normalkraft} steht immer \textbf{senkrecht} auf der Kontaktfl\"ache.}}
\\\\
\fbox{\parbox{\linewidth}{Die \textbf{Reibungskraft} zeigt immer \textbf{parallel} zu Kontaktfl\"ache.}}
\newline
\newline
\textbf{Haftreibungskraft:}\\
\[
	\boxed{F_{Zug}=F_{HR}\leq\mu_{HR}\cdot F_N}
\]
\newline
Die Haftreibung muss \"uberwunden werden, damit sich der K\"orper in Bewegung setzt.\\
Solange die angreifende Kraft $F_{Zug}$ nicht gr\"osser als $F_{HR,max}=\mu_{HR}\cdot F_N$ ist, ist die Haftreibungskraft gleich der Zugkraft.\\ 
\\
\textbf{Gleitreibung:}\\
\[
	\boxed{F_{GR}=\mu_{GR}\cdot F_N}
\]
\newline
Die Gleitreibung zwischen festen K\"orpern h\"angt nicht von deren relativer Geschwindigkeit $v$ ab.\\
\\
$\mu_{GR}<\mu_{HR}$!\\
\begin{figure}[htbp]
	\centering
	\begin{gnuplot}[scale=0.75]
		set terminal epslatex color
		set size ratio -1
		set grid
		set xrange [0:70]
		set yrange [0:35]
		set xtics 10
		set ytics 5
		set ylabel '$F$ [N] / $a$ [$\mathrm{\frac{m}{s^2}}$]'
		set style line 1 linecolor rgb 'red' linetype 1 linewidth 2
		set style line 2 linecolor rgb 'green' linetype 1 linewidth 2
		set style line 3 linecolor rgb 'blue' linetype 1 linewidth 2
		set style line 4 linecolor rgb 'orange' linetype 1 linewidth 2
		
		# define the function
		f_z(x)=x/2
		f_hr(x)=(f_z(x)<25)?f_z(x):0
		f_gr(x)=(f_z(x)<25)?0:10
		a(x)=(f_z(x)<25)?0:((f_z(x)-f_gr(x))/1.33)
		
		plot f_z(x) ti "$F_{Zug}$" with lines linestyle 1, f_hr(x) ti "$F_{HR}$" with lines linestyle 2, f_gr(x) ti "$F_{GR}$" with lines linestyle 3, a(x) ti "$a$" with lines linestyle 4
	\end{gnuplot}
\end{figure}
\subsection{Luftwiderstand}
Im Gegensatz zur Reibung zwischen festen K\"orpern ist der Luftwiderstand von der Fahrgeschwindigkeit $v$ abh\"angig.\\
\\
\begin{footnotesize}
	\begin{tabular}{|l|l}		
			\cline{0-0}
			$F_{LW,l}=b\cdot v$		& langsam, kleines v\\
			$F_{LW,s}=c\cdot v^2$	&	schnell, grosses v\\
			\cline{0-0}
	\end{tabular}
\end{footnotesize}

\section{Kurvenkr\"afte}
\subsection{Zentripetalkraft}
\[
	\boxed{F_{rad}=F_{Zentripetal}=m\cdot a_{rad}=m\frac{v^2}{R}=m\omega^2R=m\frac{4\pi^2R}{T^2}}
\]
\subsection{Neigungswinkel}
Bei h\"angenden Massenen:
\[
	\boxed{\tan\beta=\frac{\omega^2R}{g}=\frac{R}{H}}
\]
mit $H=\frac{g}{\omega^2}$; die H\"ohe unter der Aufh\"angung ist einzig eine Funktion der Kreisfrequenz $\omega$ und $g$, nicht der Seill\"ange L!
\newline
\newline
Steilwandkurven:
\[
	\boxed{\tan\beta=\frac{F_{ZP}}{F_g}=\frac{v^2}{R\cdot g}}
\]
\newline
\newline
In die Kurve liegen (Winkel $\beta$ gegen\"uber horizontalen sonst wie bei Steilwandkurve):
\[
	\boxed{\tan\beta=\frac{F_g}{F_{ZP}}=\frac{R\cdot g}{v^2}}
\]
