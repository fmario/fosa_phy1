% coding:utf-8

%----------------------------------------
%FOSAPHY, a LaTeX-Code for a summary of basic physics
%Copyright (C) 2013, Mario Felder

%This program is free software; you can redistribute it and/or
%modify it under the terms of the GNU General Public License
%as published by the Free Software Foundation; either version 2
%of the License, or (at your option) any later version.

%This program is distributed in the hope that it will be useful,
%but WITHOUT ANY WARRANTY; without even the implied warranty of
%MERCHANTABILITY or FITNESS FOR A PARTICULAR PURPOSE.  See the
%GNU General Public License for more details.
%----------------------------------------

\chapter{Schwingungen}

\section{Einfache harmonische Schwingung}
Die rücktreibende Kraft ist proportional zur Auslenkung. In diesem Fall ist die Schwingung harmonisch, d.h. eine Sinus- bzw. Kosinus-Schwingung.

\[
	-kx = F_x = ma = m \ddot{x} \rightarrow \ddot{x} + \frac{k}{m}x = 0
\]
\\
Die Lösung aus dieser homogenen Differenzialgleichung:
\[\boxed{
	x(t) = A \cos(\omega t \pm \phi)
}\]
\[\boxed{
	\omega = 2 \pi f = \sqrt{\frac{k}{m}}
}\]
\[\boxed{
	T = 2 \pi \sqrt{\frac{m}{k}}
}\]
\\
\begin{footnotesize}
	Eine positive Phase $\phi$ bedeutet eine Verschiebung nach links, ein negatives $\phi$ eine nach rechts!
\end{footnotesize}
\\
\\
Phase und Amplitude aus Anfangsbedingungen:
\[\boxed{
	\phi = \arctan \left( - \frac{v_0}{\omega x_0} \right)\qquad 
	A = \sqrt{{x_0}^2+\frac{{v_0}^2}{\omega^2}}
}\]



\subsection{$x(t)$, $v(t)$ und $a(t)$ der EHS}
Einfache harmonische Schwingung:
\[
\boxed{\begin{aligned}
	x(t) &= A \cos \left( \omega t \pm \phi \right) \\
	\dot{x}(t) &= v(t) = -\omega A \sin \left( \omega t \pm \phi \right) \\
	\ddot{x}(t) &= a(t) = -\omega^2 A \cos \left( \omega t \pm \phi \right)
\end{aligned}}\]
\\
\begin{footnotesize}
	$\omega$ = Kreisfrequenz ($2 \pi f$) \\
	$A$ = Amplitude \\
	$\phi$ = Phasenwinkel
\end{footnotesize}
\\
\\\\
$v$ aus Amplitude und Position:
\[
	v(t) = \sqrt{\frac{k}{m}} \cdot \sqrt{A^2-x^2(t)}
\]
\[
	v_{max} = \omega A = A\sqrt{\frac{k}{m}}
\]
\\
$a$ aus Amplitude und Position:
\[
	a(t) = -\frac{k}{m} \cdot x(t)
\]
\[
	a_{max} = -\omega^2\cdot A
\]
\subsection{EHS und Kreisbewegung}
Position:
\[
	r(t) = (x(t),y(t)) = A \cdot (\cos(\omega t), \sin(\omega t))
\]
\\
Geschwindigkeit:
\[
	v = \frac{\di r}{\di t} = A \omega \cdot (-\sin(\omega t), \cos(\omega t))
\]
\\
Beschleunigung:
\[
	a = \frac{v^2}{r} = A \cdot \omega^2
\]


\subsection{Energie der harmonischen Schwingung}
Gesamte energie der EHS:
\[\boxed{
	\frac12 mv^2 + \frac12kx^2 = \frac12kA^2 = \frac12m\omega^2 A^2 = \frac12 m (v_{max})^2
}\]


\subsection{Torsion, Koordinate $\theta$}
Das rücktreibende Drehmoment $M_A$ ist proportional zum Torsionswinkel $\theta$:
\[
	M_A = -\kappa \cdot \theta
\]
\\
harmonische Torsionsschwingung:
\[
	M = I_A \cdot \ddot{\theta} + \kappa \cdot \theta
\]
\[
	\theta(t) = \theta_{max} \cos(\omega t \pm \phi)
\]
\\
Kreisfrequenz und Periode:
\[\boxed{
	\omega = \sqrt{\frac{\kappa}{I_A}} \qquad T = 2 \pi \sqrt{\frac{I_A}{\kappa}}
}\]
\\
für kleine Winkel:
\[
	\kappa = k_1 \cdot {r_1}^2 + k_2 \cdot {r_2}^2 = \sum_{1}^{n}(k_n \cdot {r_n}^2)
\]


\subsection{Fadenpendel}

Es gilt die Näherung: $\sin \theta \approx \theta$ (in rad)
\[
	m\ddot{x} = F_{tan} = -mg\sin\theta \approx -mg\theta = -mg\frac{x}{L} \rightarrow \ddot{x} + \frac{g}{L}x = 0
\]
\\
Kreisfrequenz und Periodendauer:
\[\boxed{
	\omega = \sqrt{\frac{g}{L}} \qquad T = 2\pi \sqrt{\frac{L}{g}}
}\]
\\
Fadenkraft:
\[
	F_s = mg\cdot \cos\theta + m\frac{v^2}{L} = mg\cdot \cos\theta + m\omega^2 L
\]


\subsection{Physikalisches Pendel}
Anstelle von $F=ma$ wird $M=I\alpha$ verwendet:
\[
	I_z \ddot{\theta} = M_z = -d \cdot mg\sin\theta \approx -mgd\cdot \theta \rightarrow \ddot{\theta} + \frac{mgd}{I_z} \theta = 0
\]
\\
Kreisfrequenz und Periodendauer:
\[\boxed{
	\omega = \sqrt{\frac{mgd}{I_z}} \qquad T = 2\pi \sqrt{\frac{I_z}{mgd}}
}\]
\\
\begin{footnotesize}
	$d$: Abstand der Drehachse zum \textbf{Schwerpunkt}\\
	$I_z$: Trägheitsmoment bzgl. der Drehachse\\
	$m$: Masse des Körpers
\end{footnotesize}


\section{Gedämpfte Schwingung}
Reale Schwingungen sind gedämpft durch Reibung. Die Reibungskraft ist proportional zur Geschwindigkeit $v$.
\[
	\vec{F}_{Res} = ma \rightarrow \vec{F}_{Rück} + \vec{F}_{Dämpf} = -kx -bv = ma
\]

\[\boxed{
	\ddot{x} + \frac{k}{m} x + \frac{b}{m} \dot{x} = 0
}\]

\[
	F_{Dämpf} = 6\pi \cdot \eta \cdot R \cdot v \ \rightarrow \ b = 6\pi \cdot \eta \cdot R
\]
\\
\\
Radikant: $\delta^2 = \beta^2 - \omega^2$
\\\\
\textbf{1. Fall: $\delta^2 > 0$} $\rightarrow$ Kriechfall\\
allgeimeine Lösung:
\[
	x(t) = \e^{-\beta t} \left( C_1 \e^{\delta t} + C_2 \e^{-\delta t} \right)
\]
\\\\
\textbf{2. Fall: $\delta^2 = 0$} $\rightarrow$ kritische Dämpfung\\
allgeimeine Lösung:
\[
	x(t) = \e^{-\beta t} \left( C_1 t + C_2 \right)
\]
\[
	b_{krit} = \sqrt{4k\cdot m}
\]
\\\\
\textbf{3. Fall: $\delta^2 < 0$} $\rightarrow$ gedämpfte Schwingung\\
allgemeine Lösung:
\[\boxed{
	x(t)=A \cdot \e^{-\beta t} \cdot \cos(\omega_d \cdot t \pm \phi)
}\]
\\
Kreisfrequenz:
\[\boxed{
	\omega_d = \sqrt{\omega^2 - \beta^2} = \sqrt{\frac{k}{m} - \left( \frac{b}{2m} \right)^2}
}\]
\[
	\beta=\frac{b}{2m} \qquad \omega = \sqrt{\frac{k}{m}}
\]
\[
	\beta^2 < \omega^2 \ \Rightarrow \ b^2 < 4k \cdot m
\]

\subsection{Abklingkonstante $\beta$, Zerfallszeit $\tau$}
Die Amplitude zerfällt exponentiell:
\[\boxed{
	A(t) = A \e^{-\beta t} = A \e^{-\frac{t}{\tau}}
}\]
\[
	\beta = \frac{1}{\tau} \qquad\qquad \beta = \frac{\ln \left( \frac{x(t_1)}{x(t_2)} \right)}{t_2-t_1}
\]


\subsection{Torsionsschwingung mit Dämpfung}
Differenzialgleichung:
\[
	-\kappa\theta-B\dot{\theta} = I_A \ddot{\theta}
\]
\[
	\theta(t) = \theta_0 \cdot \e^{-\beta t} \cdot \cos(\omega_d t \pm \phi)
\]
\\
Daraus folgt:
\[
	\beta = \frac{B}{2I_A} \qquad \omega = \sqrt{\frac{\kappa}{I_A}} \qquad \omega_d = \sqrt{\frac{\kappa}{I_A} - \frac{B^2}{4{I_A}^2}}
\]
\\
\begin{footnotesize}
	$I_{A}$: Trägheitsmoment\\
	$\kappa$: Hookesche Prportionalitätskonstante \\
	$B$: viskose Dämpfungskonstante
\end{footnotesize}


\subsection{Physikalisches Pendel mit Dämpfung}
Differenzialgleichung:
\[
	-mgd\cdot\theta-b^*\dot{\theta} = I_z \ddot{\theta}
\]
\[
	\theta(t) = \theta_0 \cdot \e^{-\beta t} \cdot \cos(\omega_d t \pm \phi)
\]
\\
Daraus folgt:
\[
	\beta = \frac{b}{2I_z} \qquad \omega = \sqrt{\frac{mgd}{I_A}} \qquad \omega_d = \sqrt{\frac{mgd}{I_A} - \frac{b^2}{4{I_z}^2}}
\]
\\
\begin{footnotesize}
	$I_{zA}$: Trägheitsmoment\\
	$b^*$: viskose Dämpfung
\end{footnotesize}


\subsection{Elektrischer Schwingkreis}
Differenzialgleichung:
\[
	L\ddot{q} + R\dot{q} + \frac{q}{C} = 0
\]
\\
Daraus folgt:
\[
	\beta = \frac{R}{2L} \qquad \omega = \sqrt{\frac{1}{L \cdot C}} \qquad \omega_d = \sqrt{\frac{1}{L \cdot C} - \frac{R^2}{4{L}^2}}
\]
\\
\[\begin{aligned}
	Q&=\sqrt{\frac{L}{R^2 \cdot C}} \\
	q(t) &= \hat{q}\cdot \e^{-\beta t}\cdot \cos(\omega_d \cdot t + \phi) \\
	u(t) &= \frac{\hat{q}}{C} \cdot \e^{-\beta t}\cdot \cos(\omega_d \cdot t + \phi) \\
	i(t) &= \hat{q} \cdot \e^{-\beta t}\left(-\beta \cdot \cos(\omega_d \cdot t + \phi) - \omega_d \cdot \sin(\omega_d \cdot t + \phi) \right)
\end{aligned}\]
\\
\begin{footnotesize}
	$i=\difrac{q}{t}$: El. Strom\\
	$u_c = \frac{q}{C}$: Kondensatorspannung\\
	$u_L = L\difrac{i}{t}$: Spulenspannung\\
	$u_R = R \cdot i$: Spannung am Widerstand
\end{footnotesize}



\subsection{Energieverlust durch Dämpfung}
Energie am schwingenden System:
\[
	E(t) = \frac12 kx^2 + \frac12 mv^2
\]
\\
Momentante Energieänderungsrate:
\[
	\difrac{E}{t} = kx \difrac{x}{t}+ mv \difrac{v}{t} = \dot{x}(kx + m\ddot{x}) = \dot{x}\cdot F_D = -bv^2
\]
\\
Mittlere Energie des schwingenden Systems:
\[\boxed{
	\left\langle E(t) \right\rangle = \frac12 kA^2 \cdot \e^{-2\frac{t}{\tau}} = E_0 \cdot \e^{-2\frac{t}{\tau}} = E_0 \cdot \e^{-2\beta t} 
}\]


\subsection{Güte, $Q$-Faktor bei kleiner Dämpfung}
Der Energieverlust pro Zyklus wird mit der Güte ausgedrückt. Definition:
\[\boxed{
	Q = \frac{2\pi \cdot E(t)}{\left| \Delta E(t)_T \right|}
	  = \frac{2\pi \cdot E(t)}{\left| \difrac{E}{t} \cdot T \right|}
	  = \frac{\omega_d \cdot E(t)}{\left| \difrac{E}{t} \right|}
	  = \frac{\pi}{\beta \cdot T}
	  = \pi\frac{\tau}{T}
	  = \frac{\omega_d \cdot \tau}{2}
}\]
\begin{footnotesize}
	$\Delta E(t)_T$: Energieverlust pro Zyklus
\end{footnotesize}
\\\\
\begin{footnotesize}
	Je kleiner die Dämpfung $\beta$, bzw. $b$ und je grösser die Kreisfrequenz $\omega_d \approx \omega$, desto grösser die Güte $Q$.
\end{footnotesize}
\\\\
Für grosse $Q$ ($Q>5$):
\[\boxed{
	\omega_d \approx \frac{\omega}{\sqrt{1 + \frac{1}{4Q^2}}} \approx \omega \left( 1 - \frac{1}{8Q^2} \right)
}\]


\section{Erzwungene Schwingung}
Differenzialgleichung:
\[
	\ddot{x} + 2\beta \dot{x} + \omega^2x = \omega^2 H \cos(\Omega \cdot t) = \frac{F_0}{m} \cos(\Omega \cdot t)
\]
Die Amplitude $A$ und die Phase $\phi$ sind nun Funktion der Anreger-Kreisfrequenz $\Omega$:
\[\boxed{\begin{aligned}
	A(\Omega) &= \frac{F_0}{m \cdot \sqrt{\left(\omega^2 -        \Omega^2\right)^2 + \left(2\beta \cdot \Omega\right)^2}}
	          = \frac{H}{\sqrt{\left(1- \left(\frac{\Omega}{\omega}\right)^2\right)^2+\frac{b^2}{k\cdot m} \left( \frac{\Omega}{\omega} \right)^2}}\\
	&\approx \frac{H}{\sqrt{\left(1- \left(\frac{\Omega}{\omega}\right)^2\right)^2+\left( \frac{\Omega}{Q\cdot\omega} \right)^2}}
\end{aligned}}\]
\begin{footnotesize}
	$\approx$ gilt für $Q>5$
\end{footnotesize}
\\
\begin{footnotesize}
	$F_0$: Anregerkraft ($=k \cdot H$)\\
	$\Omega$: Anreger-Kreisfrequenz\\
	$H$: Anreger-Auslenkung
\end{footnotesize}


\[\boxed{
	\phi(\Omega) = \arctan \left( \frac{2\beta \cdot \Omega}{\omega^2 - \Omega^2} \right)
		= \arctan \left( \frac{\frac{b}{\sqrt{k \cdot m}} \left( \frac{\Omega}{\omega} \right)}{1 - \left(\frac{\Omega}{\omega}\right)^2 } \right)
}\]
\\
Weiterhin gilt:
\[
	\omega = \sqrt{\frac{k}{m}} \qquad
	\beta = \frac{b}{2m} = \frac{1}{\tau} \qquad
	Q = \frac{\omega_d}{2\beta} = \pi \frac{\tau}{T}
\]
Allerdings ist jetzt:
\[\boxed{
	x(t) = A(\omega) \cdot \cos(\Omega \cdot t - \phi(\omega))
}\]



\subsection{Frequenzgang}
Kurvendiskussion von $A(\Omega)$ und $\phi(\Omega)$:
\[\begin{matrix}
	\Omega := 0: & A(\Omega) = H & \phi(\Omega) = 0 \\ 
	\Omega := \infty: & A(\Omega) = 0 & \phi(\Omega) = \pi \\ 
	\Omega := \omega: & A(\Omega) = \frac{k \cdot H}{b \cdot \omega} & \phi(\Omega) = \frac{\pi}{2}
\end{matrix} \]


\subsection{Resonanz}
Resonanz bedeutet maximale Amplitude.
\[\boxed{\begin{aligned}
	\Omega_R &= \sqrt{\omega^2 - \frac{b^2}{2m^2}}
	         = \sqrt{\omega^2 - 2\beta^2}
	         \approx \omega \cdot \sqrt{1 - \frac{1}{2Q^2}}\\
	         &\approx \omega \cdot \left( 1 - \frac{1}{4Q^2} \right)
\end{aligned}}\]
\[\boxed{
	A_R = A(\Omega_R) \approx \frac{Q \cdot H}{\sqrt{1 - \frac{1}{2Q^2}}} \approx Q \cdot H
}\]
\[\boxed{
	\phi_R = \phi(\Omega_R) \approx \arctan \left( \sqrt{4Q^2 - 2} \right) \approx \frac{\pi}{2}
}\]
\begin{footnotesize}
	$\approx$ gilt für $Q>5$
\end{footnotesize}

\subsubsection{$Q$-Faktor und Resonanzkurve}
Die Güte ist ein Mass für die Peak-Schärfe. Für grosse $Q$ (=kleine Dämpfung) gilt:
\[\boxed{
	Q = \frac{\Omega_R}{\Delta \Omega}
}\]
\begin{footnotesize}
	$\Delta \Omega$: Kurfenbreite auf der Höhe $\frac{A_R}{\sqrt2}$
\end{footnotesize}